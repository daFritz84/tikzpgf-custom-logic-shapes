% Copyright 2008 by Mark Wibrow
% Copyright 2018 by Stefan Seifried
%
% This file may be distributed and/or modified
%
% 1. under the LaTeX Project Public License and/or
% 2. under the GNU Public License.
%
% See the file doc/generic/pgf/licenses/LICENSE for more details.

\usepgflibrary{shapes.gates.logic}
\usepgflibrary{shapes.gates.logic.IEC}

\pgfkeys{/pgf/.cd,
	halfsub gate IEC symbol/.initial={$HS$},
	mux gate IEC symbol/.initial={$MUX$},
	dmux gate IEC symbol/.initial={$DMUX$},
	dmux ctrlinputs/.initial=2,
	rslatch symbol/.initial={},
	dflipflop symbol/.initial={},
}

%%
%% arrange variable inputs at gates top border
%%
\def\pgf@lib@sh@logicgate@IEC@inputtopanchor#1{%
	\dimensions%
	\centerpoint%
	\advance\pgf@y\halfheight%
	\expandafter\ifx\expandafter\pgf@lib@sh@itext\csname input-#1\endcsname%
	\advance\pgf@y\invertedradius%
	\advance\pgf@y\outerinvertedradius%
	\fi%
	%
	\pgfutil@tempdima\halfwidth%
	\multiply\pgfutil@tempdima2\relax%
	\c@pgf@counta\numinputs%
	\advance\c@pgf@counta1\relax%
	\divide\pgfutil@tempdima\c@pgf@counta%
	\multiply\pgfutil@tempdima#1\relax%
	\advance\pgf@x-\halfwidth%
	\advance\pgf@x\pgfutil@tempdima%
}

%%
%% create input anchors for mux
%%
\def\pgf@lib@sh@logicgate@IEC@inputmuxanchor#1{%
	\dimensions%
	\centerpoint%
	\advance\pgf@x-\halfwidth%
	\expandafter\ifx\expandafter\pgf@lib@sh@itext\csname input-#1\endcsname%
	\advance\pgf@x-\invertedradius%
	\advance\pgf@x-\outerinvertedradius%
	\fi%
	%
	\pgfutil@tempdima\halfheight%
	%\multiply\pgfutil@tempdima2\relax%
	\c@pgf@counta\numinputs%
	\advance\c@pgf@counta1\relax%
	\divide\pgfutil@tempdima\c@pgf@counta%
	\multiply\pgfutil@tempdima#1\relax%
	%\advance\pgf@y\halfheight%
	\advance\pgf@y-\pgfutil@tempdima%
}

%%
%% create select anchors for mux
%%
\def\pgf@lib@sh@logicgate@IEC@selectmuxanchor#1{%
	\dimensions%
	\centerpoint%
	\advance\pgf@x-\halfwidth%
	\expandafter\ifx\expandafter\pgf@lib@sh@itext\csname select-#1\endcsname%
	\advance\pgf@x-\invertedradius%
	\advance\pgf@x-\outerinvertedradius%
	\fi%
	%
	\pgfutil@tempdima\halfheight%
	\c@pgf@counta\numinputs%
	\advance\c@pgf@counta1\relax%
	\divide\pgfutil@tempdima\c@pgf@counta%
	\multiply\pgfutil@tempdima#1\relax%
	\advance\pgf@y\halfheight%
	\advance\pgf@y-\pgfutil@tempdima%
}

%%
%% create select anchors for dmux
%%
\def\pgf@lib@sh@logicgate@IEC@selectdmuxanchor#1{%
	\dimensions%
	\centerpoint%
	\advance\pgf@x-\halfwidth%
	\expandafter\ifx\expandafter\pgf@lib@sh@itext\csname select-#1\endcsname%
	\advance\pgf@x-\invertedradius%
	\advance\pgf@x-\outerinvertedradius%
	\fi%
	%
	\pgfutil@tempdima\halfheight%
	\c@pgf@counta\ctrlinputs%
	\advance\c@pgf@counta1\relax%
	\divide\pgfutil@tempdima\c@pgf@counta%
	\multiply\pgfutil@tempdima#1\relax%
	\advance\pgf@y\halfheight%
	\advance\pgf@y-\pgfutil@tempdima%
}

%%
%% create output anchors for dmux
%%
\def\pgf@lib@sh@logicgate@IEC@outputanchor#1{%
	\dimensions%
	\centerpoint%
	\advance\pgf@x+\halfwidth%
	\expandafter\ifx\expandafter\pgf@lib@sh@itext\csname output-#1\endcsname%
	\advance\pgf@x-\invertedradius%
	\advance\pgf@x-\outerinvertedradius%
	\fi%
	%
	\pgfutil@tempdima\halfheight%
	\multiply\pgfutil@tempdima2\relax%
	\c@pgf@counta\numoutputs%
	\advance\c@pgf@counta1\relax%
	\divide\pgfutil@tempdima\c@pgf@counta%
	\multiply\pgfutil@tempdima#1\relax%
	\advance\pgf@y\halfheight%
	\advance\pgf@y-\pgfutil@tempdima%
}

% Shape halfsub
%
\pgfdeclareshape{halfsub}{
	\expandafter\pgfutil@g@addto@macro\csname pgf@sh@s@halfsub\endcsname{%
		\pgf@lib@sh@logicgate@parseinputs{2}%
		%
		\pgfmathloop%
		\ifnum\pgfmathcounter>\pgf@lib@sh@logicgate@numinputs%
		\else%
		\pgfutil@ifundefined{pgf@anchor@halfsub@input \pgfmathcounter}{%
			\expandafter\xdef\csname pgf@anchor@halfsub@input \pgfmathcounter\endcsname{%
				\noexpand\pgf@lib@sh@logicgate@IEC@inputtopanchor{\pgfmathcounter}%
			}%
		}{}%
		\repeatpgfmathloop%
		\ifnum\pgf@lib@sh@logicgate@numinputs=2\relax%
		\else%
			\pgferror{An halfsub gate must have exactly two inputs}%
		\fi%
	}%
	\savedmacro\numinputs{\let\numinputs\pgf@lib@sh@logicgate@numinputs}%
	\saveddimen\invertedradius{%
		\pgfmathsetlength\pgf@x{\pgfkeysvalueof{/pgf/logic gate inverted radius}}%
	}%
	\saveddimen\outerinvertedradius{%
		\pgfmathsetlength\pgf@x{\pgfkeysvalueof{/pgf/logic gate inverted radius}}%
		\advance\pgf@x.5\pgflinewidth%
	}
	\savedmacro\dimensions{%
		\pgf@lib@sh@logicgates@dimensions@IEC{halfsub}%
	}
	\savedanchor\centerpoint{%
		\pgf@x.5\wd\pgfnodeparttextbox%
		\pgf@y.5\ht\pgfnodeparttextbox%
		\advance\pgf@y-.5\dp\pgfnodeparttextbox%
	}
	\savedanchor\midpoint{%
		\pgf@x.5\wd\pgfnodeparttextbox%
		\pgfmathsetlength\pgf@y{+0.5ex}%
	}
	\savedanchor\basepoint{%
		\pgf@x.5\wd\pgfnodeparttextbox%
		\pgf@y0pt%
	}

	%%
	%% standard anchors inherited from and gate
	%%
	\inheritanchor[from=and gate IEC]{center}
	\inheritanchor[from=and gate IEC]{mid}
	\inheritanchor[from=and gate IEC]{mid west}
	\inheritanchor[from=and gate IEC]{mid east}
	\inheritanchor[from=and gate IEC]{base}
	\inheritanchor[from=and gate IEC]{base west}
	\inheritanchor[from=and gate IEC]{base east}
	\inheritanchor[from=and gate IEC]{north}
	\inheritanchor[from=and gate IEC]{south}
	\inheritanchor[from=and gate IEC]{east}
	\inheritanchor[from=and gate IEC]{west}
	\inheritanchor[from=and gate IEC]{north east}
	\inheritanchor[from=and gate IEC]{north west}
	\inheritanchor[from=and gate IEC]{south east}
	\inheritanchor[from=and gate IEC]{south west}
	
	%%
	%% define anchor for difference and carry
	%%
	\anchor{diff}{\dimensions\centerpoint\advance\pgf@y-\halfheight}
	\anchor{carry}{\dimensions\centerpoint\advance\pgf@x-\halfwidth}
	
	%%
	%% draw surrounding rectangle
	%%
	\backgroundpath{%
		\dimensions%
		\pgf@xc\halfwidth%
		\pgf@yc\halfheight%
		\pgfmathaddtolength\pgf@xc{-\pgfkeysvalueof{/pgf/outer xsep}}%
		\pgfmathaddtolength\pgf@yc{-\pgfkeysvalueof{/pgf/outer xsep}}%
		{%
			\pgftransformshift{\centerpoint}%
			\pgfpathrectangle{\pgfqpoint{-\pgf@xc}{-\pgf@yc}}%
			{%
				\pgf@xc2.0\pgf@xc%
				\pgf@yc2.0\pgf@yc%
				\pgfqpoint{\pgf@xc}{\pgf@yc}%
			}%
			\pgfpathclose%
			%
			% Draw inputs.
			%
			\pgf@yc+\halfheight%
			\advance\pgf@yc+\invertedradius%
			\pgf@xc\halfwidth%
			\pgfutil@tempdima2.0\pgf@xc%
			\c@pgf@counta\numinputs%
			\advance\c@pgf@counta1\relax%
			\divide\pgfutil@tempdima\c@pgf@counta%
			\pgfmathloop%
			\ifnum\pgfmathcounter>\numinputs%
			\else%
			\advance\pgf@xc-\pgfutil@tempdima%
			\expandafter\ifx\expandafter\pgf@lib@sh@itext\csname input-\pgfmathcounter\endcsname%
			{%
				\pgfpathcircle{\pgfqpoint{\pgf@xc}{\pgf@yc}}{\invertedradius}%      
			}%
			\fi%
			\repeatpgfmathloop%
		}%
	}
	
	%%
	%% inherit foreground from and gate
	%%
	\foregroundpath{\pgf@lib@sh@logicgates@IEC@foregroundpath{halfsub}}
	\inheritanchorborder[from=and gate IEC]
}

% Shape mux
%
\pgfdeclareshape{mux}{
	\expandafter\pgfutil@g@addto@macro\csname pgf@sh@s@mux\endcsname{%
		\pgf@lib@sh@logicgate@parseinputs{1024}% Maximum 1024 (!) inputs.
		%
		\pgfmathloop%
		\ifnum\pgfmathcounter>\pgf@lib@sh@logicgate@numinputs%
		\else%
		\pgfutil@ifundefined{pgf@anchor@mux@input \pgfmathcounter}{%
			\expandafter\xdef\csname pgf@anchor@mux@input \pgfmathcounter\endcsname{%
				\noexpand\pgf@lib@sh@logicgate@IEC@inputmuxanchor{\pgfmathcounter}%
			}%
		}{}%
		\repeatpgfmathloop%
		%
		\pgfmathparse{log2(\pgf@lib@sh@logicgate@numinputs+1)}%
		\edef\pgf@lib@sh@logicgate@selinputs{\pgfmathresult}%
		\pgfmathloop%
		\ifnum\pgfmathcounter>\pgf@lib@sh@logicgate@selinputs%
		\else%
		\pgfutil@ifundefined{pgf@anchor@mux@select \pgfmathcounter}{%
			\expandafter\xdef\csname pgf@anchor@mux@select \pgfmathcounter\endcsname{%
				\noexpand\pgf@lib@sh@logicgate@IEC@selectmuxanchor{\pgfmathcounter}%
			}%
		}{}%
		\repeatpgfmathloop%
		%
		\pgfmathparse{Mod(\pgf@lib@sh@logicgate@numinputs,2)==0?1:0}
		\ifnum\pgfmathresult=0\relax%
		\pgferror{A mux gate must have mod 2 inputs}%
		\fi%
	}%
	\savedmacro\numinputs{\let\numinputs\pgf@lib@sh@logicgate@numinputs}%
	\saveddimen\invertedradius{%
		\pgfmathsetlength\pgf@x{\pgfkeysvalueof{/pgf/logic gate inverted radius}}%
	}%
	\saveddimen\outerinvertedradius{%
		\pgfmathsetlength\pgf@x{\pgfkeysvalueof{/pgf/logic gate inverted radius}}%
		\advance\pgf@x.5\pgflinewidth%
	}
	\savedmacro\dimensions{%
		\pgf@lib@sh@logicgates@dimensions@IEC{mux}%
	}
	\savedanchor\centerpoint{%
		\pgf@x.5\wd\pgfnodeparttextbox%
		\pgf@y.5\ht\pgfnodeparttextbox%
		\advance\pgf@y-.5\dp\pgfnodeparttextbox%
	}
	\savedanchor\midpoint{%
		\pgf@x.5\wd\pgfnodeparttextbox%
		\pgfmathsetlength\pgf@y{+0.5ex}%
	}
	\savedanchor\basepoint{%
		\pgf@x.5\wd\pgfnodeparttextbox%
		\pgf@y0pt%
	}
	
	%%
	%% standard anchors inherited from and gate
	%%
	\inheritanchor[from=and gate IEC]{center}
	\inheritanchor[from=and gate IEC]{mid}
	\inheritanchor[from=and gate IEC]{mid west}
	\inheritanchor[from=and gate IEC]{mid east}
	\inheritanchor[from=and gate IEC]{base}
	\inheritanchor[from=and gate IEC]{base west}
	\inheritanchor[from=and gate IEC]{base east}
	\inheritanchor[from=and gate IEC]{north}
	\inheritanchor[from=and gate IEC]{south}
	\inheritanchor[from=and gate IEC]{east}
	\inheritanchor[from=and gate IEC]{west}
	\inheritanchor[from=and gate IEC]{north east}
	\inheritanchor[from=and gate IEC]{north west}
	\inheritanchor[from=and gate IEC]{south east}
	\inheritanchor[from=and gate IEC]{south west}
	
	%%
	%% define anchor for output
	%%
	\anchor{output}{\dimensions\centerpoint\advance\pgf@x\halfwidth}
	
	%%
	%% draw surrounding rectangle
	%%
	\backgroundpath{%
		\dimensions%
		\pgf@xc\halfwidth%
		\pgf@yc\halfheight%
		\pgfmathaddtolength\pgf@xc{-\pgfkeysvalueof{/pgf/outer xsep}}%
		\pgfmathaddtolength\pgf@yc{-\pgfkeysvalueof{/pgf/outer xsep}}%
		{%
			\pgftransformshift{\centerpoint}%
			\pgfpathrectangle{\pgfqpoint{-\pgf@xc}{-\pgf@yc}}%
			{%
				\pgf@xc2.0\pgf@xc%
				\pgf@yc2.0\pgf@yc%
				\pgfqpoint{\pgf@xc}{\pgf@yc}%
			}%
			\pgfpathclose%
			%
			% Draw inputs.
			%
			\pgf@xc-\halfwidth%
			\advance\pgf@xc-\invertedradius%
			\pgf@yc-\halfheight%
			\pgfutil@tempdima1.0\pgf@yc%
			\c@pgf@counta\numinputs%
			\advance\c@pgf@counta1\relax%
			\divide\pgfutil@tempdima\c@pgf@counta%
			\pgfmathloop%
			\ifnum\pgfmathcounter>\numinputs%
			\else%
			\advance\pgf@yc-\pgfutil@tempdima%
			\expandafter\ifx\expandafter\pgf@lib@sh@itext\csname input-\pgfmathcounter\endcsname%
			{%
				\pgfpathcircle{\pgfqpoint{\pgf@xc}{\pgf@yc}}{\invertedradius}%      
			}%
			\fi%
			\repeatpgfmathloop%
		}%
	}
	
	%%
	%% inherit foreground from and gate
	%%
	\foregroundpath{\pgf@lib@sh@logicgates@IEC@foregroundpath{mux}}
	\inheritanchorborder[from=and gate IEC]
}

% Shape dmux
%
\pgfdeclareshape{dmux}{
	\expandafter\pgfutil@g@addto@macro\csname pgf@sh@s@dmux\endcsname{%
		\pgf@lib@sh@logicgate@parseinputs{1024}% Maximum 1024 (!) inputs.
		%
		% parse number of inputs
		%
		\pgfmathloop%
		\ifnum\pgfmathcounter>\pgf@lib@sh@logicgate@numinputs%
		\else%
		\pgfutil@ifundefined{pgf@anchor@dmux@input \pgfmathcounter}{%
			\expandafter\xdef\csname pgf@anchor@dmux@input \pgfmathcounter\endcsname{%
				\noexpand\pgf@lib@sh@logicgate@IEC@inputmuxanchor{\pgfmathcounter}%
			}%
		}{}%
		\repeatpgfmathloop%
		%
		% get number of control/select inputs
		%
		\edef\pgf@lib@sh@logicgate@ctrlinputs{\pgfkeysvalueof{/pgf/dmux ctrlinputs}}%
		\pgfmathloop%
		\ifnum\pgfmathcounter>\pgf@lib@sh@logicgate@ctrlinputs%
		\else%
		\pgfutil@ifundefined{pgf@anchor@dmux@select \pgfmathcounter}{%
			%\message{adding anchor \pgfmathcounter ^^J}% debug msg
			\expandafter\xdef\csname pgf@anchor@dmux@select \pgfmathcounter\endcsname{%
				\noexpand\pgf@lib@sh@logicgate@IEC@selectdmuxanchor{\pgfmathcounter}%
			}%
		}{}%
		\repeatpgfmathloop%
		%
		% create outputs based on number of ctrlinputs and inputs
		%
		\pgfmathparse{pow(\pgf@lib@sh@logicgate@ctrlinputs,2)*\pgf@lib@sh@logicgate@numinputs}%
		\edef\pgf@lib@sh@logicgate@numoutputs{\pgfmathresult}%
		%\message{number of outputs \pgf@lib@sh@logicgate@numoutputs ^^J}% debug msg
		\pgfmathloop%
		\ifnum\pgfmathcounter>\pgf@lib@sh@logicgate@numoutputs%
		\else%
		\pgfutil@ifundefined{pgf@anchor@dmux@output \pgfmathcounter}{%
			\expandafter\xdef\csname pgf@anchor@dmux@output \pgfmathcounter\endcsname{%
				\noexpand\pgf@lib@sh@logicgate@IEC@outputanchor{\pgfmathcounter}%
			}%
		}{}%
		\repeatpgfmathloop%
		
	}%
	\savedmacro\numinputs{\let\numinputs\pgf@lib@sh@logicgate@numinputs}%
	\savedmacro\ctrlinputs{\let\ctrlinputs\pgf@lib@sh@logicgate@ctrlinputs}%
	\savedmacro\numoutputs{\let\numoutputs\pgf@lib@sh@logicgate@numoutputs}%
	\saveddimen\invertedradius{%
		\pgfmathsetlength\pgf@x{\pgfkeysvalueof{/pgf/logic gate inverted radius}}%
	}%
	\saveddimen\outerinvertedradius{%
		\pgfmathsetlength\pgf@x{\pgfkeysvalueof{/pgf/logic gate inverted radius}}%
		\advance\pgf@x.5\pgflinewidth%
	}
	\savedmacro\dimensions{%
		\pgf@lib@sh@logicgates@dimensions@IEC{dmux}%
	}
	\savedanchor\centerpoint{%
		\pgf@x.5\wd\pgfnodeparttextbox%
		\pgf@y.5\ht\pgfnodeparttextbox%
		\advance\pgf@y-.5\dp\pgfnodeparttextbox%
	}
	\savedanchor\midpoint{%
		\pgf@x.5\wd\pgfnodeparttextbox%
		\pgfmathsetlength\pgf@y{+0.5ex}%
	}
	\savedanchor\basepoint{%
		\pgf@x.5\wd\pgfnodeparttextbox%
		\pgf@y0pt%
	}
	
	%%
	%% standard anchors inherited from and gate
	%%
	\inheritanchor[from=and gate IEC]{center}
	\inheritanchor[from=and gate IEC]{mid}
	\inheritanchor[from=and gate IEC]{mid west}
	\inheritanchor[from=and gate IEC]{mid east}
	\inheritanchor[from=and gate IEC]{base}
	\inheritanchor[from=and gate IEC]{base west}
	\inheritanchor[from=and gate IEC]{base east}
	\inheritanchor[from=and gate IEC]{north}
	\inheritanchor[from=and gate IEC]{south}
	\inheritanchor[from=and gate IEC]{east}
	\inheritanchor[from=and gate IEC]{west}
	\inheritanchor[from=and gate IEC]{north east}
	\inheritanchor[from=and gate IEC]{north west}
	\inheritanchor[from=and gate IEC]{south east}
	\inheritanchor[from=and gate IEC]{south west}
		
	%%
	%% draw surrounding rectangle
	%%
	\backgroundpath{%
		\dimensions%
		\pgf@xc\halfwidth%
		\pgf@yc\halfheight%
		\pgfmathaddtolength\pgf@xc{-\pgfkeysvalueof{/pgf/outer xsep}}%
		\pgfmathaddtolength\pgf@yc{-\pgfkeysvalueof{/pgf/outer xsep}}%
		{%
			\pgftransformshift{\centerpoint}%
			\pgfpathrectangle{\pgfqpoint{-\pgf@xc}{-\pgf@yc}}%
			{%
				\pgf@xc2.0\pgf@xc%
				\pgf@yc2.0\pgf@yc%
				\pgfqpoint{\pgf@xc}{\pgf@yc}%
			}%
			\pgfpathclose%
			%
			% Draw inputs.
			%
			\pgf@xc-\halfwidth%
			\advance\pgf@xc-\invertedradius%
			\pgf@yc-\halfheight%
			\pgfutil@tempdima1.0\pgf@yc%
			\c@pgf@counta\numinputs%
			\advance\c@pgf@counta1\relax%
			\divide\pgfutil@tempdima\c@pgf@counta%
			\pgfmathloop%
			\ifnum\pgfmathcounter>\numinputs%
			\else%
			\advance\pgf@yc-\pgfutil@tempdima%
			\expandafter\ifx\expandafter\pgf@lib@sh@itext\csname input-\pgfmathcounter\endcsname%
			{%
				\pgfpathcircle{\pgfqpoint{\pgf@xc}{\pgf@yc}}{\invertedradius}%      
			}%
			\fi%
			\repeatpgfmathloop%
		}%
	}
	
	%%
	%% inherit foreground from and gate
	%%
	\foregroundpath{\pgf@lib@sh@logicgates@IEC@foregroundpath{dmux}}
	\inheritanchorborder[from=and gate IEC]
}

% Shape rslatch
% Inspired by the work of Martin Scharrer
\pgfdeclareshape{rslatch}{
	% The 'minimum width' and 'minimum height' keys, not the content, determine
	% the size
	\savedanchor\northeast{%
		\pgfmathsetlength\pgf@x{\pgfshapeminwidth}%
		\pgfmathsetlength\pgf@y{\pgfshapeminheight}%
		\pgf@x=0.5\pgf@x
		\pgf@y=0.5\pgf@y
	}
	% This is redundant, but makes some things easier:
	\savedanchor\southwest{%
		\pgfmathsetlength\pgf@x{\pgfshapeminwidth}%
		\pgfmathsetlength\pgf@y{\pgfshapeminheight}%
		\pgf@x=-0.5\pgf@x
		\pgf@y=-0.5\pgf@y
	}
	% Inherit from rectangle
	\inheritanchorborder[from=rectangle]
	
	% Define same anchor a normal rectangle has
	\anchor{center}{\pgfpointorigin}
	\anchor{north}{\northeast \pgf@x=0pt}
	\anchor{east}{\northeast \pgf@y=0pt}
	\anchor{south}{\southwest \pgf@x=0pt}
	\anchor{west}{\southwest \pgf@y=0pt}
	\anchor{north east}{\northeast}
	\anchor{north west}{\northeast \pgf@x=-\pgf@x}
	\anchor{south west}{\southwest}
	\anchor{south east}{\southwest \pgf@x=-\pgf@x}
	\anchor{text}{
		\pgfpointorigin
		\advance\pgf@x by -.5\wd\pgfnodeparttextbox%
		\advance\pgf@y by -.5\ht\pgfnodeparttextbox%
		\advance\pgf@y by +.5\dp\pgfnodeparttextbox%
	}
	
	% Define anchors for signal ports
	\anchor{R}{
		\pgf@process{\northeast}%
		\pgf@x=-1\pgf@x%
		\pgf@y=0.5\pgf@y%
	}
	\anchor{S}{
		\pgf@process{\northeast}%
		\pgf@x=-1\pgf@x%
		\pgf@y=-0.5\pgf@y%
	}
	\anchor{Q}{
		\pgf@process{\northeast}%
		\pgf@y=.5\pgf@y%
	}
	\anchor{Qn}{
		\pgf@process{\northeast}%
		\pgf@y=-.5\pgf@y%
	}

	% Draw the rectangle box and the port labels
	\backgroundpath{
		% Rectangle box
		\pgfpathrectanglecorners{\southwest}{\northeast}
		
		% Draw port labels
		\begingroup
		\tikzset{font=\sffamily\scriptsize} % Use font from this style
		\tikz@textfont
		
		\pgf@anchor@rslatch@R
		\pgftext[left,base,at={\pgfpoint{\pgf@x}{\pgf@y}},x=\pgfshapeinnerxsep]{\raisebox{-0.75ex}{R}}
		
		\pgf@anchor@rslatch@S
		\pgftext[left,base,at={\pgfpoint{\pgf@x}{\pgf@y}},x=\pgfshapeinnerxsep]{\raisebox{-0.75ex}{S}}
		
		\pgf@anchor@rslatch@Q
		\pgftext[right,base,at={\pgfpoint{\pgf@x}{\pgf@y}},x=-\pgfshapeinnerxsep]{\raisebox{-.75ex}{Q}}
		
		\pgf@anchor@rslatch@Qn
		\pgftext[right,base,at={\pgfpoint{\pgf@x}{\pgf@y}},x=-\pgfshapeinnerxsep]{\raisebox{-.75ex}{$\overline{\mbox{Q}}$}}
		
		\endgroup
	}
}

% Shape dflipflop
% Inspired by the work of Martin Scharrer
\pgfdeclareshape{dflipflop}{
	% The 'minimum width' and 'minimum height' keys, not the content, determine
	% the size
	\savedanchor\northeast{%
		\pgfmathsetlength\pgf@x{\pgfshapeminwidth}%
		\pgfmathsetlength\pgf@y{\pgfshapeminheight}%
		\pgf@x=0.5\pgf@x
		\pgf@y=0.5\pgf@y
	}
	% This is redundant, but makes some things easier:
	\savedanchor\southwest{%
		\pgfmathsetlength\pgf@x{\pgfshapeminwidth}%
		\pgfmathsetlength\pgf@y{\pgfshapeminheight}%
		\pgf@x=-0.5\pgf@x
		\pgf@y=-0.5\pgf@y
	}
	% Inherit from rectangle
	\inheritanchorborder[from=rectangle]
	
	% Define same anchor a normal rectangle has
	\anchor{center}{\pgfpointorigin}
	\anchor{north}{\northeast \pgf@x=0pt}
	\anchor{east}{\northeast \pgf@y=0pt}
	\anchor{south}{\southwest \pgf@x=0pt}
	\anchor{west}{\southwest \pgf@y=0pt}
	\anchor{north east}{\northeast}
	\anchor{north west}{\northeast \pgf@x=-\pgf@x}
	\anchor{south west}{\southwest}
	\anchor{south east}{\southwest \pgf@x=-\pgf@x}
	\anchor{text}{
		\pgfpointorigin
		\advance\pgf@x by -.5\wd\pgfnodeparttextbox%
		\advance\pgf@y by -.5\ht\pgfnodeparttextbox%
		\advance\pgf@y by +.5\dp\pgfnodeparttextbox%
	}
	
	% Define anchors for signal ports
	\anchor{D}{
		\pgf@process{\northeast}%
		\pgf@x=-1\pgf@x%
		\pgf@y=.5\pgf@y%
	}
	\anchor{CLK}{
		\pgf@process{\northeast}%
		\pgf@x=-1\pgf@x%
		\pgf@y=-0\pgf@y%
	}
	\anchor{Q}{
		\pgf@process{\northeast}%
		\pgf@y=.5\pgf@y%
	}
	\anchor{Qn}{
		\pgf@process{\northeast}%
		\pgf@y=-.5\pgf@y%
	}

	% Draw the rectangle box and the port labels
	\backgroundpath{
		% Rectangle box
		\pgfpathrectanglecorners{\southwest}{\northeast}
	    % Angle (>) for clock input
		\pgf@anchor@dflipflop@CLK
		\pgf@xa=\pgf@x \pgf@ya=\pgf@y
		\pgf@xb=\pgf@x \pgf@yb=\pgf@y
		\pgf@xc=\pgf@x \pgf@yc=\pgf@y
		\pgfmathsetlength\pgf@x{1.6ex} % size depends on font size
		\advance\pgf@ya by 0.5\pgf@x
		\advance\pgf@xb by \pgf@x
		\advance\pgf@yc by -0.5\pgf@x
		\pgfpathmoveto{\pgfpoint{\pgf@xa}{\pgf@ya}}
		\pgfpathlineto{\pgfpoint{\pgf@xb}{\pgf@yb}}
		\pgfpathlineto{\pgfpoint{\pgf@xc}{\pgf@yc}}
		\pgfclosepath
		
		% Draw port labels
		\begingroup
		\tikzset{font=\sffamily\scriptsize} % Use font from this style
		\tikz@textfont
		
		\pgf@anchor@dflipflop@D
		\pgftext[left,base,at={\pgfpoint{\pgf@x}{\pgf@y}},x=\pgfshapeinnerxsep]{\raisebox{-0.75ex}{D}}
		
		\pgf@anchor@dflipflop@CLK
		\pgftext[left,base,at={\pgfpoint{\pgf@x+2.0ex}{\pgf@y}},x=\pgfshapeinnerxsep]{\raisebox{-0.75ex}{C}}
		
		
		\pgf@anchor@dflipflop@Q
		\pgftext[right,base,at={\pgfpoint{\pgf@x}{\pgf@y}},x=-\pgfshapeinnerxsep]{\raisebox{-.75ex}{Q}}
		
		\pgf@anchor@dflipflop@Qn
		\pgftext[right,base,at={\pgfpoint{\pgf@x}{\pgf@y}},x=-\pgfshapeinnerxsep]{\raisebox{-.75ex}{$\overline{\mbox{Q}}$}}
		\endgroup
	}
}

\endinput